\documentclass{article}

\usepackage{hyperref}

% Below will configure hyperlink colors
\hypersetup{
    colorlinks=true,
    linkcolor=blue,
    filecolor=blue,
    urlcolor=blue,
}
\urlstyle{same}

\usepackage{enumitem}

% Below package is for precise positioning of images
\usepackage{float}
\usepackage{color}

% Below package is for syntax highlighted code
\usepackage{minted}
% Below set global setting fro minted package like line wraps and frame
\setminted{breaklines, frame=single}

% Below package is for including images
\usepackage{graphicx}
\usepackage[a4paper, inner=1.5cm, outer=3cm, top=2cm,bottom=3cm, bindingoffset=1cm]{geometry}
\setlength{\parskip}{.5em}

% The LaTeX graphics/graphicx package uses the first dot to find the extension. Package grffile changes this algorithm. This helps in including images files whose name contains more than 1 dot
\usepackage{grffile}


% Below package will create headings
%\pagestyle{headings}

% Below package will used to specify captions like "Figure 1: Nice Figure" in minipages using \captionof{figure}{Nice Figure}
\usepackage{caption}
\usepackage{hypcap} %This is needed for some warning in minipages


\begin{document}
\begin{titlepage}
   \vspace*{\stretch{1.0}}
   \begin{center}
      \Large\textsc{C Unsigned/Signed Integer Arithmetic 101}\\
      \vspace{5mm}
      \Large\textit{Vineel Kovvuri}\\
      \url{http://vineelkovvuri.com}\\
   \end{center}
   \vspace*{\stretch{2.0}}
\end{titlepage}

\tableofcontents

\newpage
\section{Introduction}
This article is about understanding how integer conversions happen in C
language. C standard defines the integer conversion rules not specific to any
architecure. This makes things more complicated for the programmers to try and
understand them.

Why do we need integer conversions at all? The answer is simple we need to have
one single type for the expression that we are dealing with. Let's say we have
an expression \textless expr1\textgreater \textless op\textgreater
\textless expr2\textgreater  when expr1 and expr2 are different types, we
want the resulting expression from this to have one type.

Before going any further in to the details of the arthemetic conversions. Let's
look at few basic concepts. The C99 specification refers to below types as
standard integer types.

\begin{enumerate}[noitemsep]
    \item signed char
    \item unsigned char
    \item signed short
    \item unsigned short
    \item signed int
    \item unsigned int
    \item signed long
    \item unsigned long
    \item signed long long
    \item unsigned long long
\end{enumerate}

One of the interesting fact about C language specification is, the sizes of
these integer types is not defined at all. It only defines the minimum supported
length. For example, int need not be 8 bytes long on x64 platforms, the only
definition for int is it should have atleast 16 bits and similarly long should
have atleast 32 bits it need not be 8 bytes long. Depending on the platform
and the processor architecture the ABI(Application Binary Interface) and the
64bit programming model will define the size of these basic types. Windows x64
follow what is called as LLP64(meaning only 'long long' and pointer size is 64
bit wide), So this is what we are sticking to in this article.

\begin{center}
\begin{tabular}{|c|c|c|}
 Type       &Size       \\
 \hline               
 signed char         & 1 bytes \\
 unsigned char       & 1 bytes \\
 signed short        & 2 bytes \\
 unsigned short      & 2 bytes \\
 signed int          & 4 bytes \\
 unsigned int        & 4 bytes \\
 signed long         & 4 bytes \\
 unsigned long       & 4 bytes \\
 signed long long    & 8 bytes \\
 unsigned long long  & 8 bytes
\end{tabular}
\end{center}

Also, The specification leaves other aspects of C langauge definition undefined
and this made room for optimization for the compiler of the choosen platform.
For example, the result of signed arithemetic leading to overflow/underflow is
not defined because this predates to the machine architectures when 2's
complement representation for -ve numbers is not universal, Even though
virutally every architecture now uses 2's complement representation for -ve
numbers. Hence result of unsigned overflow/underflow is well defined but not
signed overflow/underflow!

\section{How signedness is represented in the hardware?}
Before we understand what are signed and unsigned integers, we need to
understand weather the processor has this notion of signed and unsigned
integers concept. Interestingly enough processors do have the concept of
signed/unsigned but unlike in C language, where this information is baked in to
the variable type definition, processor registers do not hold this type
information. After all, registers are just place holders for any kind of
data. But the instructions themself have the notion of signed/unsigned, That
is why we have signed 'imul', unsigned 'mul', signed(JL/JG) vs unsigned(JB/JA)
jump instructions. This is an important distinction between programming
languages and machine code. It helps in understanding how high level code is get
translated to underlying machine code. Instructions which produce/consume signed
data understand that -ve numbers should be in 2's complement form. So if a
register(8bit) has 11110110 its up to the instruction to treat that data as
either -10 (signed) or 246 (unsinged). If an operation results in a -ve result
say -5 then destination register will be written 11111011(2's complement
representation of -5). 'add' and 'sub' instructions themself are not effected
by signed and unsigned numbers because of modulo arithmetic.

\section{Signed vs Unsigned integers in C}
Signed numbers are the way -ve numbers are represented and unsigned numbers are
the way in which +ve numbers are represented. If lets say a data type has 1
byte of storage then the possible bit representation will be from  0b00000000(0)
- 0b11111111(255)




\section{References}
\begin{enumerate}[noitemsep]
\item \href{https://wiki.sei.cmu.edu/confluence/display/c/INT02-C.+Understand+integer+conversion+rules}{INT02-C. Understand integer conversion rules}
\item \href{http://www.open-std.org/jtc1/sc22/wg14/www/docs/n1256.pdf}{C99 Standard}
\item \href{http://blog.robertelder.org/signed-or-unsigned/}{Should I use Signed or Unsigned Ints In C? (Part 1)}
\item \href{http://blog.robertelder.org/signed-or-unsigned-part-2/}{Should I use Signed or Unsigned Ints In C? (Part 2)}
\item \href{https://stackoverflow.com/a/42589535/2407966}{Why is imul used for multiplying unsigned numbers?}

\end{enumerate}

\end{document}