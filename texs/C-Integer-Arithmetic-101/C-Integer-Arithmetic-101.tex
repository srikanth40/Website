\documentclass{article}

\usepackage{hyperref}

% Below will configure hyperlink colors
\hypersetup{
    colorlinks=true,
    linkcolor=blue,
    filecolor=blue,
    urlcolor=blue,
}
\urlstyle{same}

\usepackage{enumitem}

% Below package is for precise positioning of images
\usepackage{float}
\usepackage{color}

% Below package is for syntax highlighted code
\usepackage{minted}
% Below set global setting fro minted package like line wraps and frame
\setminted{breaklines, frame=single}

% Below package is for including images
\usepackage{graphicx}
\usepackage[a4paper, inner=1.5cm, outer=3cm, top=2cm,bottom=3cm, bindingoffset=1cm]{geometry}
\setlength{\parskip}{.5em}

% The LaTeX graphics/graphicx package uses the first dot to find the extension. Package grffile changes this algorithm. This helps in including images files whose name contains more than 1 dot
\usepackage{grffile}


% Below package will create headings
%\pagestyle{headings}

% Below package will used to specify captions like "Figure 1: Nice Figure" in minipages using \captionof{figure}{Nice Figure}
\usepackage{caption}
\usepackage{hypcap} %This is needed for some warning in minipages


\begin{document}
\begin{titlepage}
   \vspace*{\stretch{1.0}}
   \begin{center}
      \Large\textsc{C Unsigned/Signed Integer Arithmetic 101}\\
      \vspace{5mm}
      \Large\textit{Vineel Kovvuri}\\
      \url{http://vineelkovvuri.com}\\
   \end{center}
   \vspace*{\stretch{2.0}}
\end{titlepage}

\tableofcontents

\newpage
\section{Introduction}
This article is about understanding how integer conversions happen in C
language. C standard defines the integer conversion rules not specific to any
architecure. This makes things more complicated for the programmers to try and
understand them.

Why do we need integer conversions at all? The answer is simple we need to have
one single type for the expression that we are dealing with. Let's say we have
an expression \textless expr1\textgreater \textless op\textgreater
\textless expr2\textgreater  when expr1 and expr2 are different types, we
want the resulting expression from this to have one type.

Before going any further in to the details of the arthemetic conversions. Let's
look at few basic concepts. The C99 specification refers to below types as
standard integer types.

\begin{enumerate}[noitemsep]
    \item signed char
    \item unsigned char
    \item signed short
    \item unsigned short
    \item signed int
    \item unsigned int
    \item signed long
    \item unsigned long
    \item signed long long
    \item unsigned long long
\end{enumerate}

One of the interesting fact about C language specification is, the sizes of
these integer types is not defined at all. It only defines the minimum supported
length. For example, int need not be 8 bytes long on x64 platforms, the only
definition for int is it should have atleast 16 bits and similarly long should
have atleast 32 bits it need not be 8 bytes long. Depending on the platform
and the processor architecture the ABI(Application Binary Interface) and the
64bit programming model will define the size of these basic types. Windows x64
follow what is called as LLP64(meaning only 'long long' and pointer size is 64
bit wide), So this is what we are sticking to in this article.

\begin{center}
\begin{tabular}{|c|c|c|}
 Type       &Size       \\
 \hline
 signed char         & 1 bytes \\
 unsigned char       & 1 bytes \\
 signed short        & 2 bytes \\
 unsigned short      & 2 bytes \\
 signed int          & 4 bytes \\
 unsigned int        & 4 bytes \\
 signed long         & 4 bytes \\
 unsigned long       & 4 bytes \\
 signed long long    & 8 bytes \\
 unsigned long long  & 8 bytes
\end{tabular}
\end{center}

Also, The specification leaves other aspects of C langauge definition undefined
and this made room for optimization for the compiler of the choosen platform.
For example, the result of signed arithemetic leading to overflow/underflow is
not defined because this predates to the machine architectures when 2's
complement representation for -ve numbers is not universal, Even though
virutally every architecture now uses 2's complement representation for -ve
numbers. Hence result of unsigned overflow/underflow is well defined but not
signed overflow/underflow!

\section{How signedness is represented in the hardware?}
Before we understand what are signed and unsigned integers, we need to
understand weather the processor has this notion of signed and unsigned
integers concept. Interestingly enough processors do have the concept of
signed/unsigned but unlike in C language, where this information is baked in to
the variable type definition, processor registers do not hold this type
information. After all, registers are just place holders for any kind of
data. But the instructions themself have the notion of signed/unsigned, That
is why we have signed 'imul', unsigned 'mul', signed(JL/JG) vs unsigned(JB/JA)
jump instructions. This is an important distinction between programming
languages and machine code. It helps in understanding how high level code is get
translated to underlying machine code. Instructions which produce/consume signed
data understand that -ve numbers should be in 2's complement form. So if a
register(8bit) has 11110110 its up to the instruction to treat that data as
either -10 (signed) or 246 (unsinged). If an operation results in a -ve result
say -5 then destination register will be written 11111011(2's complement
representation of -5). 'add' and 'sub' instructions themself are not effected
by signed and unsigned numbers because of modulo arithmetic.

\section{Signed vs Unsigned integers}
Signed numbers are the way -ve numbers are represented and unsigned numbers are
the way in which +ve numbers are represented. If lets say a data type has 1
byte of storage then the possible bit representation will be from  0b00000000(0)
- 0b11111111(255). Now these 255 valid slots can be interpretted as all positive
numbers or divide the range in two half and call one half of the slot as
positive and other as negative numbers. There are many ways in which this
division can be made but we use 2's complement notation to represent -ve
numbers. The take away here is if the 255 slots are represented by -ve numbers
then we will represent from -128 to 127. According to 2's complement notation
-128 is represented as 0b10000000(0x80) where as 127 is represented as
0b01111111(0x7F).

<hand draw the picture for number line and interpose positive and negative
numbers>

<Discuss a bit about how signed flag overflags gets set and how signed and
unsigned jumps work in asm>


\section{Integer Arithmetic Conversion in C}
C language defines below set of rules to convert the arguments in an expression.

\subsection{Rank}
Every integer type has an integer conversion rank defined as follows:
\begin{enumerate}[noitemsep]
    \item No two signed integer types shall have the same rank, even if they have the same
    representation.
    \item The rank of a signed integer type shall be greater than the rank of any signed integer
    type with less precision.
    \item The rank of long long int shall be greater than the rank of long int, which
    shall be greater than the rank of int, which shall be greater than the rank of short
    int, which shall be greater than the rank of signed char.
    \item The rank of any unsigned integer type shall equal the rank of the corresponding
    signed integer type, if any.
    \item The rank of any standard integer type shall be greater than the rank of any extended
    integer type with the same width.
    \item The rank of char shall equal the rank of signed char and unsigned char.
    \item The rank of \_Bool shall be less than the rank of all other standard integer types.
    \item The rank of any enumerated type shall equal the rank of the compatible integer type
    (see 6.7.2.2).
    \item The rank of any extended signed integer type relative to another extended signed
    integer type with the same precision is implementation-defined, but still subject to the
    other rules for determining the integer conversion rank.
    \item For all integer types T1, T2, and T3, if T1 has greater rank than T2 and T2 has
    greater rank than T3, then T1 has greater rank than T3.
\end{enumerate}

\subsection{Integer Promotion}
    The following may be used in an expression wherever an int or unsigned int may
    be used:
    \begin{itemize}[noitemsep]
        \item An object or expression with an integer type whose integer conversion rank is less
        than or equal to the rank of int and unsigned int.
        \item A bit-field of type \_Bool, int, signed int, or unsigned int.
    \end{itemize}
\begin{itemize}[noitemsep]
    \item 1a) If an int can represent all values of the original type, the value is converted to an int;
    \item 1b) otherwise, it is converted to an unsigned int.
    These are called the integer promotions. All other types are unchanged by the integer promotions.
\end{itemize}

\subsection{Usual arithmetic conversions}
\begin{itemize}[noitemsep]
    \item 2a) If both operands have the same type, then no further conversion is needed.

    \item 2b) Otherwise, if both operands have signed integer types or both have unsigned
    integer types, the operand with the type of lesser integer conversion rank is
    converted to the type of the operand with greater rank.

    \item 2c) Otherwise, if the operand that has unsigned integer type has rank greater or
    equal to the rank of the type of the other operand, then the operand with
    signed integer type is converted to the type of the operand with unsigned
    integer type.

    \item 2d) Otherwise, if the type of the operand with signed integer type can represent
    all of the values of the type of the operand with unsigned integer type, then
    the operand with unsigned integer type is converted to the type of the
    operand with signed integer type.

    \item 2e) Otherwise, both operands are converted to the unsigned integer type
    corresponding to the type of the operand with signed integer type.
\end{itemize}

The above rules need some explanation. The gist of 2c is make sense of small -ve
number with large postive number. If for example a 4 byte signed int is added
to 8 byte unsigned long long, according to this rule 4 byte signed int will
become 8 byte unsigned long long. The important point to remember here is
converion of signed int to unsigned long long, The way the compiler does this
sign extend 4 byte signed number so that its absolute value will not change
i.e.., -10 remain -10 whether it is a 4 byte or 8 byte. But according to these
rules the expression will have the type unsigned long long so statments like
below will result in unexpected behaviour. Since the var1+var2 is an unsigned
expression the compiler will enforce it by using a unsigned jump instruction
called jae instead of jge instruction.






\begin{minted}{c}
signed int var1 = -100;
unsigned long long var2 = 10;
if (var1 + var2 < 0) { //resulting unsigned long long expression will never be lessthan 0
    printf("This will never get printed");
}
\end{minted}
\begin{minted}{asm}
$LN4:
        sub     rsp, 56                             ; 00000038H
        mov     DWORD PTR var1$[rsp], -100                ; ffffffffffffff9cH
        mov     QWORD PTR var2$[rsp], 10
        movsxd  rax, DWORD PTR var1$[rsp]
        add     rax, QWORD PTR var2$[rsp]
        test    rax, rax
        jae     SHORT $LN2@main             --> jae is an unsigned jump for above or equal
        lea     rcx, OFFSET FLAT:$SG4416
        call    printf
$LN2@main:
        xor     eax, eax
        add     rsp, 56                             ; 00000038H
        ret     0
main    ENDP
\end{minted}



\section{References}
\begin{enumerate}[noitemsep]
\item \href{https://wiki.sei.cmu.edu/confluence/display/c/INT02-C.+Understand+integer+conversion+rules}{INT02-C. Understand integer conversion rules}
\item \href{http://www.open-std.org/jtc1/sc22/wg14/www/docs/n1256.pdf}{C99 Standard}
\item \href{http://blog.robertelder.org/signed-or-unsigned/}{Should I use Signed or Unsigned Ints In C? (Part 1)}
\item \href{http://blog.robertelder.org/signed-or-unsigned-part-2/}{Should I use Signed or Unsigned Ints In C? (Part 2)}
\item \href{http://ref.x86asm.net/coder-abc.html}{X86 Opcode and Instruction Reference}
\item \href{https://stackoverflow.com/a/42589535/2407966}{Why is imul used for multiplying unsigned numbers?}

\end{enumerate}

\end{document}