\documentclass{article}

\usepackage{hyperref}

% Below will configure hyperlink colors
\hypersetup{
    colorlinks=true,
    linkcolor=blue,
    filecolor=blue,
    urlcolor=blue,
}
\urlstyle{same}

\usepackage{enumitem}

% Below package is for precise positioning of images
\usepackage{float}
\usepackage{color}

% Below package is for syntax highlighted code
\usepackage{minted}
% Below set global setting fro minted package like line wraps and frame
\setminted{breaklines, frame=single}

% Below package is for including images
\usepackage{graphicx}
\usepackage[a4paper, inner=1.5cm, outer=3cm, top=2cm,bottom=3cm, bindingoffset=1cm]{geometry}
\setlength{\parskip}{.5em}

% The LaTeX graphics/graphicx package uses the first dot to find the extension. Package grffile changes this algorithm. This helps in including images files whose name contains more than 1 dot
\usepackage{grffile}


% Below package will create headings
%\pagestyle{headings}

% Below package will used to specify captions like "Figure 1: Nice Figure" in minipages using \captionof{figure}{Nice Figure}
\usepackage{caption}
\usepackage{hypcap} %This is needed for some warning in minipages

% Below package will used to print < and > symbols using \textless and \textgreater
\usepackage{lmodern}

\begin{document}
\begin{titlepage}
   \vspace*{\stretch{1.0}}
   \begin{center}
      \Large\textsc{Lib Files 101}\\
      \vspace{5mm}
      \Large\textit{Vineel Kovvuri}\\
      \url{http://vineelkovvuri.com}\\
   \end{center}
   \vspace*{\stretch{2.0}}
\end{titlepage}

\newpage
\section{Introduction}
During the compilation one of the crucial step after assembling is
creating the Object files. The collection of these object files is
called a lib file. We can make these .lib files when creating below
two type of projects
1. Static Library
2. Dynamic Linked Library

The format of these .lib files is specified in 'Archive (Library) File Format.'
section of \href{https://docs.microsoft.com/en-us/windows/desktop/debug/pe-format#archive-library-file-format}{PE Format}
specification. As per the spec, .lib is an archive of individual .obj
files with some metadata. Multiple tools can be used to
extract lib files. Visual Studio installation contains Lib.exe tool.
Because .lib and .obj files follow Unix COFF format Unix binutil's
'ar' tool can be used to extract it.

\begin{minted}{bat}
Lib StaticLib1.lib /list
Lib StaticLib1.lib /EXTRACT:Debug\sub.obj /out:sub.obj <-- Extracts one file at a time!

or

ar -x StaticLib1.lib

or

use 7-Zip to extract it
\end{minted}

\section{Static Library}
Static Library is created when you want to provide the complete code to
link into another dll or exe. For example, If a static library
project contains 4 files add.c,sub.c,mul.c,div.c containing functions
for their operations like below respectively.
\begin{minted}{c}
int add(int a)
{
    return a + 1;
}
\end{minted}

When you build the project what we get is a .lib file containing obj files
of each of the above .c file. So a static library contains all
the code that gets compiled from your project, and this .lib will
be consumed by any other project types(dll or exe).

NOTE: The functions inside a Static Library is not declared with
\_\_declspec(dllexport) because all functions declared in a static
library are meant to be consumed/included by others directly.

\begin{minted}{bat}
dumpbin /exports StaticLib.lib  <-- shows nothing because .lib itself does not export anything
dumpbin /symbols StaticLib.lib  <-- shows all the symbols present
\end{minted}

\begin{figure}[H]
\centering
\includegraphics[width=\textwidth]{1.StaticLib.png}
\caption{Static Library}
\end{figure}

\begin{figure}[H]
\centering
\includegraphics{2.StructureStaticLib.png}
\caption{Concluded based on HxD view of the file}
\end{figure}

\section{Dynamic Linked Library}
Dynamic Linked Library(DLL) is in many ways similar to Static Library because
it also provides the code to be used by other projects like dll or exe,
but the difference is in the way the code gets used by them.
In DLL, the outcome of the project is not only a .lib file but also a
.dll file. In fact, in case of DLL project, the .lib file does not contain
any .obj file instead it contains only pointers to which dll includes
the code and how other projects can refer to it in their compilation.
In Dll, all of the code is indeed present inside the .dll file.

Hence this is also the reason why we can only see the exported symbols
but not their code in case of dll's .lib file.

\begin{minted}{bat}
dumpbin /exports Dll1.lib <-- shows all the exported functions
\end{minted}

\begin{figure}[H]
\centering
\includegraphics[width=\textwidth]{3.Dll.png}
\caption{Dynamic Linked Library}
\end{figure}
\begin{figure}[H]
\centering
\includegraphics{4.StructureDll.png}
\caption{Concluded based on HxD view of the file}
\end{figure}

NOTE: The functions inside a Dynamic Linked Library need to be declared
with \_\_declspec(dllexport) if they need to be visible and consumed
by others(indirectly).

\section{References}
\begin{enumerate}[noitemsep]
\item \href{https://docs.microsoft.com/en-us/windows/desktop/debug/pe-format}{PE Format}
\end{enumerate}

\end{document}